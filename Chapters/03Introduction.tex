
\section{Motivation}

Since the publication of the Bitcoin~\cite{bitcoin} whitepaper by the pseudonym {\tt Satoshi Nakamoto} in 2008, the concept of blockchain has gained widespread use in various domains. The technology has evolved and has become more sophisticated, allowing for more complex use cases than just preventing double spending in money transfers. This was allowed by the use of Smart Contracts, immutable\footnote{The theoretical immutability of smart contracts is analyzed more in-depth in chapter 6.} and deterministic pieces of code that rule the outcome of transactions based on the logic written in the code.

Smart Contracts were first implemented in 2015 by the Ethereum blockchain~\cite{Ethereum}, the first of many \textit{Blockchain 2.0} that allowed developers to build decentralized applications. 

Ethereum is a permissionless blockchain, it can be seen as a digital public ledger. It can be modified just through append operations and is immutable. One of the main strengths of this kind of blockchain is the fact that the public ledger is transparent, so everyone can independently download and access and verify the data.

Reading and understanding this public ledger can bring huge value, since it describes the entire history of a market that, as of now, is valued hundreds of billions of dollars. Apart from the economical aspects, the knowledge of on-chain data is an essential building block for Decentralised Applications (dApps), that are, by definition, applications which interact with Smart Contracts.

Unfortunately, as noted also in other works~\cite{dataether,xblock-eth,ethereum_query_language}, extracting and analyzing data from the Ethereum blockchain is not an easy task. The reason is that the amount of data is huge and Ethereum nodes store it for optimizing storage instead of the ease of access. 

It is possible to query the Ethereum nodes by just a few parameters, such as transaction hashes, block numbers and indexed log topics. 
Without an external index, it is impossible to search data based on any other attribute.

My work aims to ease access to on-chain data and describes a way to do it using \textit{Dgraph}~\cite{dgraph}, an open-source distributed database.

\section{Research questions}

Two research questions were defined:

\begin{itemize}
    \item \textbf{RQ1}: What kind of information is possible to extract from EVM blockchains without relying on centralized services?
    \item \textbf{RQ2}: What computational resources are required to independently extract and index the entire history of the Ethereum blockchain in August 2023?
\end{itemize} 

\section{Contribution}

This thesis provides multiple contributions to the topic of data mining from the Ethereum blockchain:

\begin{itemize}
    \item A research on the main state-of-the-art tools, highlighting their strengths and shortcomings.
    \item The definition of a schema for Ethereum data optimized for graph databases. This schema includes both raw data and the semantics that can be built from it.
    \item The release of \textbf{eth2dgraph}, an open-source software written in \textit{Rust} that efficiently extracts data from the Ethereum blockchain to be indexed and queried with \textit{Dgraph}, a distributed graph database. 
    \item An analysis of the Ethereum data, both in terms of infrastructure and time needed to perform extraction and indexing and in terms of what semantics can be extracted.
\end{itemize}

\section{Outline}

The next chapters of this thesis as structured as follows:

\begin{itemize}
    \item \hyperref[chapter-2]{\textbf{Chapter 2 - Background}}: this chapter introduces the technical details of how a blockchain works, with a focus on Ethereum. It also includes a section about Dgraph. 
    \item \hyperref[chapter-3]{\textbf{Chapter 3 - Previous work}}: in this chapter, each section describes a work done in the field of data extraction and/or indexing of blockchain data.
    \item \hyperref[chapter-4]{\textbf{Chapter 4 - Methods}}: here it is described in details how eth2dgraph works. It is shown how each piece of information has been extracted.
    \item \hyperref[chapter-5]{\textbf{Chapter 5 - Results}}: this chapter shows the outcome of running eth2dgraph using a local Ethereum node to extract and index all the history of the chain.  
    \item \hyperref[chapter-analysis]{\textbf{Chapter 6 - Analysis of data}}: in this chapter there are six independent analysis on the data extracted that show the state of the chain.
    \item \hyperref[chapter-discussion]{\textbf{Chapter 7 - Discussion}}: In this chapter are discussed the results of this research, the future work and more generally the upcoming challenges of the sector.
    \item \hyperref[chapter-conclusions]{\textbf{Chapter 8 - Conclusions}}: the last chapter summarizes the main findings and gives answers to the research questions
\end{itemize}

